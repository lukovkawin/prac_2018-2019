\documentclass{beamer}
\usepackage[utf8]{inputenc}
\usepackage[T1]{fontenc}
\usepackage[russian]{babel}

\usetheme{focus}

\usepackage{booktabs}
\usepackage{gensymb}

\title{Практикум на ЭВМ\\ Мечи залива работорговцев}

\subtitle{Задание 2}

\author{Кригер Ксения \\ Семакова Екатерина \\ Тангаев Артем}

\titlegraphic{\includegraphics[width=4cm]{targ2.PNG}}

\institute{МГУ ВМК \\ Исследование Операций}

\date{\today}

\begin{document}

    \scriptsize
    
    \begin{frame}
        \maketitle
    \end{frame}
    
    \begin{frame}{Постановка задачи}
        Два поставщика стали -- это компании {Westeros Inc.} и {Harpy \& Co}. Каждая из них предлагает ощутимую скидку при заключении эксклюзивного договора на поставку.\\
        Требуется принять объективное решение вопроса о том, с какой из компаний следует заключить договор на поставку стали.\\
        Имеются записи о производстве мечей каждым из кузнецов-безупречных, а также данные о количестве сломанных мечей в каждый из месяцев ведения боевых действий.\\
    \end{frame}
    
    \begin{frame}{Цель работы}
        Провести разведывательный анализ, и выявить более выгодного поставщика.\\
        \vspace{0.5cm}
        Разведывательный анализ данных — анализ основных свойств данных, нахождение в них общих закономерностей, распределений и аномалий, построение начальных моделей. \\
    \end{frame}
    
    \begin{frame}{Анализ}
        Были вычислены и оценены следующие метрики:\\
        \begin{itemize}
            \item Количество произведенных и сломанных мечей
            \item Срок службы мечей
            \item Стабильность поставок
        \end{itemize}
    \end{frame}
    
    \begin{frame}{Количество}
        \begin{figure}
            \centering
            \includegraphics[width=8cm]{graph2.jpg}
        \end{figure}
        
        \begin{table}[H]
            \begin{right}
                \begin{tabular}{|c|c|c|}
                    \hline
                    Поставщик & Произведено & Сломано \\
                    \hline
                    Westeros Inc. & 31625 & 8268 \\
                    \hline
                    Harpy \& Co & 31523 & 6080 \\
                    \hline
                \end{tabular}
            \end{right}
        \end{table} 
        
        При почти одинаковом количестве произведенных мечей, сломанных заметно больше среди мечей сделанных из стали Westeros Inc.
    \end{frame}
    
    \begin{frame}{Стабильность поставок}
        \begin{figure}
            \centering
            \includegraphics[width=8cm]{graph1.jpg}
        \end{figure}
        Количество произведенных мечей в месяц. У кузнецов, работающих со сталью Harpy \& Co производство мечей более стабильное, возможно, это связано с поставками и качеством стали. 
    \end{frame}
    
    \begin{frame}{Срок службы}
        \begin{figure}
            \centering
            \includegraphics[width=8cm]{graph3.jpg}
        \end{figure}
        Мечи из стали Harpy \& Co в первые три месяца почти не ломаются, то есть для коротких войн данный поставщик выгоднее.\\
        Количество сломанных мечей из стали Westeros Inc. в течение всего срока эксплуатации находится на одном уровне.
        Еще одно очко в пользу южан!
    \end{frame}
    
    \begin{frame}{Срок службы}
        \begin{figure}
            \centering
            \includegraphics[width=8cm]{graph4.jpg}
        \end{figure}
        Из этого графика видно, что мечи из стали Westeros Inc. начинают активно ломаться с первого месяца использования. \\
        В то время как, мечи из стали Harpy \& Co превосходно показывают себя в первые три месяца. Но и после пика поломок в четвертом месяце, количество дефектов резко снижается, что даст возсожность завершить войну.
    \end{frame}
    
    \begin{frame}{Качество партии}
        \begin{figure}
            \centering
            \includegraphics[width=8cm]{graph5.jpg}
        \end{figure}
        Здесь оценивается количество оставшихся целых мечей после оцениваемого срока использования.\\
        Видно, что начиная со второго месяца в каждой партии остаетсябольше целых мечей из стали Harpy \& Co чем мечей из стали Westeros Inc.
    \end{frame}
    
    \begin{frame}{Вывод}
        \begin{figure}
            \centering
            \includegraphics[width=8cm]{garp.jpg}
        \end{figure}
        На основе этих данных видно, что не важно, ведете ли вы короткую войну в несколько месяцев или планируете долгосрочное сражение в несколько лет, выгоднее закупать сталь у поставщика Harpy \& Co.
    \end{frame}

\end{document}