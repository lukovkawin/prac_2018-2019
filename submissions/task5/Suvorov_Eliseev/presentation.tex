\setbeamertemplate{caption}[numbered]
\setbeamertemplate{navigation symbols}{}
\setbeamercolor*{block title}{fg=chameleongreen3}
\setbeamercolor*{palette quaternary}{fg=white, bg=chameleongreen1}
\usetheme{Berlin}
\usecolortheme{beaver}
\begin{document}
	\title{Отчет о выполнении  II задания по практикуму}  
	\author{Алескандр Суворов, Елисеев Павел}
	\institute{Московский государственный университет им. М. В. Ломоносова}
	\date{Москва, 2019} 
	\frame{\titlepage} 
	\frame{\frametitle{Постановка задачи}
		\begin{itemize}
			\item Есть два поставщика стали: {\bf Westeros Inc. и Harpy \& Co.} \\*Необходимо выбрать компанию, с которой следует заключить эксклюзивный договор на поставку стали.
			\item Необходимо провести {\bf разведывательный анализ данных} с целью ответа на вопрос: {\it "<С каким из поставщиков стали следует заключить договор?">}
		\end{itemize}\tableofcontents}
			
	\begin{frame}
		\frametitle{Исходные данные}
		\begin{itemize}
			\item Дан CSV-файл с данными о производстве оружия и количестве единиц сломанного оружия за каждый месяц каждой из компаний.
		\end{itemize}
	\end{frame}
	
	\begin{frame}
		\frametitle{Описание выполнения задания}
		\begin{itemize}
			\item<1-> Было рассмотрено несколько выборок, для сравнения качества продукции: 
			\begin{enumerate}
				\item Выборка показывает среднее число поломанной продукции после каждого месяца эксплуатации.
				\item Выборка показывает общий срок службы продукции по месяцам.
				\item Выборка показывает суммарное число сломанной продукции по месяцам.
			\end{enumerate}
		\end{itemize}
	\end{frame}
	\begin{frame}
		\frametitle{Описание выполнения задания}
		Для каждой выборки было построено две диаграммы. Сравнивая данные графического отображения выборок, можно сделать выводы о качестве продукции.
		\\*
		\textbf{Диаграмма размаха}\\
		Диаграмма показывает сразу несколько параметров распределения:\\
		1)Медиану\\
		2)Стенками ящика (квартили 0.25 и 0.75) ограничены 50\% выборки;
		Это может быть просто минимальное и максимальное значения в выборке.\\
		\textbf{Вторая диаграмма} - полторы ширины ящика (расстояния между квартилями 0.25 и 0.75). Тогда точками отмечаются выбросы. В нашей программе это просто минимальное и максимальное значения в выборке. 
	\end{frame}
	\begin{frame}[shrink=4]
		\frametitle{Описание выполнения задания}
		\framesubtitle{Выборка №1}

        \includegraphics[scale=0.5, width=6cm, height=4cm]{$1}
        \includegraphics[scale=0.5, width=6cm, height=4cm]{$2}\\
		По первым диаграмам сложно сделать какие-то выводы, т.к. в первые три месяца мечи из стали Harpy \& Co почти не ломаются, зато с 4 месяца количество сломанных мечей резко возрастает. А для мечей из стали Westeros Inc характерно, что ломается примерно одинаковое количество каждый месяц. Но при это медиана для компании Harpy \& Co ниже. По нашему мнению можно немного склонить часу весов в пользу Harpy \& Co.
	\end{frame}
	\begin{frame}[shrink=4]
		\frametitle{Описание выполнения задания}
		\framesubtitle{Выборка №2}
        \includegraphics[scale=0.5, width=6cm, height=4cm]{$3}
        \includegraphics[scale=0.5, width=6cm, height=4cm]{$4}\\
		На данных диаграммах показан важный критерий, это общий срок службы мечей, но как видно из графиков данные параметры почти не отличаются по двум компаниям, результат немного лучше у компании Westeris Inc, но это совсем незначительно, как мы видим в перспективе. 
	\end{frame}
	\begin{frame}[shrink=4]
		\frametitle{Описание выполнения задания}
		\framesubtitle{Выборка №3}
        \includegraphics[scale=0.5, width=6cm, height=4cm]{$5}
        \includegraphics[scale=0.5, width=6cm, height=4cm]{$6}\\
		На данном слайде представлен еще один важный показатель, это количество целых мечей. По данному показателю компания Harpy \& Co сильно опережает компанию Westeros Inc.
	\end{frame}
	\begin{frame}
	\frametitle{Анализ результатов}
		\begin{itemize}
			\item {Как было замечено выше, первая выборка не дает однозначной оценки при разведывательном анализе.}
			\item {Вторая выборка весьма важна при оценке продукта, но у двух компаний результаты почти одинаковы, поэтому по данному критерию сложно сделать выбор в пользу какой-либо компании. }
			\item {Третья выборка тоже показывает важный критерий, т.к. чем больше число целых мечей, тем большее число воинов могут участвовать в сражениях одновременно и тем выше численность войска. Численность войска является важным параметром при прочих равных условиях, особенно этот критерий важен для нападающей стороны.}
			\item {Можно сделать вывод, что предпочтительнее закупать сталь у компании Harpy \& Co, т.к. их сталь их сумарное число сломаной продукции наименьшее.}
		\end{itemize}
	\end{frame}
\end{document}